\chapter{How to write thesis in LaTeX\label{chap:how_to}}

\section{Versioning with git}

Write the LaTeX in such a way that it could be versioned by git, which will help when collaborating with other people.
This means writing \textbf{one sentence per line}.
Even when you use third-party platforms, such as the OverLeaf, you can still share the repository through Git.

\section{Forming paragraphs}

A paragraph is formed in LaTeX by an uninterrupted block of non-empty lines.
It is recommended to keep a single sentence per line (helps with versioning using git).
A new paragraph is started after an empty line.

This is a new paragraph. It is strongly recommended to \textbf{avoid} the use of the \emph{newline} (\texttt{\textbackslash\textbackslash}) feature of LaTeX for forming paragraphs as it doesn't format the new paragraph properly (no space at beginning of the new paragraph).

\section{Linguistic anti patterns}

\subsection{Narrative}

We recommend to write your thesis in plural form of the first-person narrative in combination with passive tense, e.g.:
\begin{itemize}
    \item We discourage the use of any other form, and/or
    \item any other form is discouraged, but \textbf{not}
    \item {\color{red} I discourage you from using the first-person narrative}.
\end{itemize}
Moreover, avoid \enquote{instructional} or \enquote{teacher}-like style of writing, such as {\color{red} \enquote{Now, we multiply the matrix $\mathbf{A}$ by the scalar $c$ to get the scaled matrix $\mathbf{B}$.}}
A better way of writing the same information would be e.g. \enquote{Now, the scaled matrix $\mathbf{B}$ is obtained by multiplying the matrix $\mathbf{A}$ by the scalar $c$.}


\subsection{Pronouns}

The use of pronouns (it, this, they) is strongly \textbf{discouraged}.
Although, pronouns make it easier for you as a writer to form the flow of the text, pronouns also make it much more difficult for the reader to follow the text.
The reader is forced to retain more of the context to substitute and understand what the author meant.
Moreover, pronouns can easily become vague (there is more than one way how to interpret them) and can become invalid while making editorial changes to the text, i.e., when moving sentences around.
A technical text should be written in a way that makes it as easy to read and comprehend as possible and as hard to misunderstand or misinterpret as possible at the same time.

\section{Mathematical notation with LaTeX}

Take care to use the correct mathematical symbols and common ways of denoting mathematical concepts.
Use bold fonts to visually distinguish vectors and matrices ($\mathbf{x}$, $\mathbf{A}$) and scalars ($k$, $N$).

\subsection{Equations}
Mathematical equations should be numbered and should be a part of a sentence.
For example, a discrete LTI system update is described as
\begin{equation}
    \mathbf{x}_{\left[k+1\right]} = \mathbf{A}\mathbf{x}_{\left[k\right]} + \mathbf{B}\mathbf{u}_{\left[k\right]},
    \label{eq:lti_system}
\end{equation}
where $\mathbf{x}_{\left[k\right]} \in \mathbb{R}^m$ is the state vector at the sample $k$, $\mathbf{u}_{\left[k\right]} \in \mathbb{R}^n$ is the input vector, $\mathbf{A} \in \mathbb{R}^{m \times m}$ is the main system matrix, and $\mathbf{B} \in \mathbb{R}^{m \times n}$ is the system input matrix.
Proper punctuation should be used after the equation, as if it were an ordinary object in the sentence.

Do not put any empty lines before the equation.
If the sentence that the equation is a part of continues after the equation (as is the case here), do not put empty lines after the equation either.
That would create a new paragraph mid-sentence.
{\color{red}
For an example of how not to do it, the equation

\begin{equation}
    \mathrm{\sigma}(x) = \frac{1}{1 + e^{-x}}
\end{equation}

describes the logistic function often used in machine learning.
}
Observe how a new paragraph is created for the equation and then for this block of text (compare with the proper typesetting above).
Not only does this not look correct, it may also cause incorrect page breaking.

\section{Using footnotes}

Do not be afraid to use footnotes for additional information, such as http links\footnote{This repository: \url{https://github.com/Abdulrasheed1729/unibadan_template}.}.
We use footnote links whenever we want to \emph{point} to a website, rather then to cite it as a source.
Like with everything, do not overdo it.

\section{Referencing document elements}

LaTeX allows you to dynamically reference to parts of the documents as shown below:
\begin{lstlisting}[caption={LaTeX macros for referencing to document elements.},label={lst:references}]
    \newcommand{\reffig}[1]{Fig.~\ref{#1}}
    \newcommand{\reflst}[1]{Lst.~\ref{#1}}
    \newcommand{\refalg}[1]{Alg.~\ref{#1}}
    \newcommand{\refsec}[1]{Sec.~\ref{#1}}
    \newcommand{\reftab}[1]{Table~\ref{#1}}
    \newcommand{\refeq}[1]{\eqref{#1}}
  \end{lstlisting}

